\documentclass{article}
\title{Comportamiento estrat\'egico}
\author{Pablo Coello\\ pcoello@gmail.com \and X\'acome Laya\\ xacomelide@gmail.com}
\date{15/02/19}
\usepackage{Sweave}
\usepackage{synttree} 
\usepackage{slashbox}
\usepackage{subfig}
\begin{document}
\maketitle
\abstract{Tareas para entregar de la asignatura An\'alisis del Comportamiento Estrat\'egico. M\'aster en Econom\'ia.}

\newpage
\tableofcontents
\newpage
\Sconcordance{concordance:PUTAPRUEBA.tex:PUTAPRUEBA.Rnw:%
1 26 1 1 2 1 0 3 1 4 0 1 2 4 1 1 2 1 0 3 1 7 0 2 2 1 0 2 1 6 0 1 2 88 1 %
1 2 1 0 1 2 1 0 3 1 7 0 1 2 2 1 1 2 1 0 3 1 7 0 1 2 138 1}

\Sconcordance{concordance:PUTAPRUEBA.tex:PUTAPRUEBA.Rnw:%
1 26 1 1 2 1 0 3 1 4 0 1 2 4 1 1 2 1 0 3 1 7 0 2 2 1 0 2 1 6 0 1 2 88 1 %
1 2 1 0 1 2 1 0 3 1 7 0 1 2 2 1 1 2 1 0 3 1 7 0 1 2 138 1}



\section{Ejercicio 1: Buscar equilibrios de Nash y poner en forma extensiva los siguientes juegos}
\subsection{Dilema del prisionero}
Forma normal:
\begin{table}[htbp]
\begin{center}
\begin{tabular}{|l|l|l|}
\hline
1/2 & C & N\\
\hline \hline
C & -3,-3 & 0,-4 \\ \hline
N & -4,0 & -1,-1 \\ \hline

\end{tabular}
\caption{Dilema del prisionero}
\label{tabla:sencilla}
\end{center}
\end{table}

Forma extensiva:
\begin{center}
\synttree[1[2[$(-3,-3)$][$(0,-4)$]][2[$(-4,0)$][$(-1,-1)$]]]
\end{center}

O, alternativamente:
\begin{center}
\synttree[2[1[$(-3,-3)$][$(0,-4)$]][1[$(-4,0)$][$(-1,-1)$]]]
\end{center}

En este juego el equilibrio de Nash es la combinaci\'on de estrategias para el jugador 1 y el jugador 2 (C,C), en la que ambos jugadores deciden confesar.

\subsection{Batalla de los sexos}
Forma normal:
\begin{table}[htbp]
\begin{center}
\begin{tabular}{|l|l|l|}
\hline
1/2 & B & C\\
\hline \hline
B & 2,1 & 0,0 \\ \hline
C & -1,-1 & 1,2 \\ \hline

\end{tabular}
\caption{Batalla de los sexos}
\label{tabla:sencilla}
\end{center}
\end{table}

Forma extensiva:
\begin{center}
\synttree[1[2[$(2,1)$][$(-1,-1)$]][2[$(0,0)$][$(1,2)$]]]
\end{center}
O, alternativamente:
\begin{center}
\synttree[2[1[$(2,1)$][$(-1,-1)$]][1[$(0,0)$][$(1,2)$]]]
\end{center}

En este juego,  existen dos equilibrios de Nash que corresponden con las combinaciones de estrategias para el jugador 1 y el jugador 2 (B,B) y (C,C), en los que ambos jugadores deciden asistir al ballet o al cine correspondientemente.

\subsection{Estudiar la forma de de representar JNC con tres jugadores}

La siguiente figura corresponde a un juego respresentado en su forma extensiva:
\begin{center}
\synttree[1[2[3[(3,1,5)][(2,6,3)]][(2,6,2)]][3[3[(4,5,0)][(3,2,5)]][1[(4,5,0)][(1,3,6)]]]]
\end{center}
Y esta es su correspondiente forma normal, en la que las estrategias del jugador uno est\'an descritas por las filas, las del jugador 3 en columnas y las estrategias del jugador 2 en matrices:

\begin{table}[htbp]
\begin{center}
\begin{tabular}{|l|l|l|l|l|}
\hline
1/3 & eg & eh & fg & fh \\
\hline \hline
al & (3,1,5) & (2,6,3) & (3,1,5) & (2,6,3) \\ \hline
ak & (3,1,5) & (2,6,3) & (3,1,5) & (2,6,3) \\ \hline
bl & (4,5,0) & (3,2,5) & (4,5,0) & (3,2,5) \\ \hline
bk & (4,5,0) & (3,2,5) & (4,5,0) & (3,2,5) \\ \hline
\end{tabular}
\caption{Estrategia c del jugador 2}
\label{tabla:sencilla}
\end{center}
\end{table}

\begin{table}[htbp]
\begin{center}
\begin{tabular}{|l|l|l|l|l|}
\hline
1/3 & eg & eh & fg & fh \\
\hline \hline
al & (2,6,2) & (2,6,2) & (2,6,2) & (2,6,2) \\ \hline
ak & (2,6,2) & (2,6,2) & (2,6,2) & (2,6,2) \\ \hline
bl & (4,5,0) & (3,2,5) & (4,5,0) & (3,2,5) \\ \hline
bk & (4,5,0) & (3,2,5) & (4,5,0) & (3,2,5) \\ \hline
\end{tabular}
\caption{Estrategia d del jugador 2}
\label{tabla:sencilla}
\end{center}
\end{table}



\section{Ejercicio 2: Juegos de Cournot}

En el siguiente Juego a la Cournot la demanda de mercado que enfrentan n empresas es estrictamente convexa con la forma
$p(q)=\frac{a}{Q^\alpha}$, donde $Q=\sum_{i=1}^n q_i$. Los consumidores se asumen pasivos y vienen descritos por $p(Q)=a-bQ$.

\subsection{Empresas producen con $C_i(q_i)=cq_i$}
Si consideramos que el n\'umero de empresas \textbf{n} es 2, podemos calcular las cantidades que maximizan sus beneficios como:
 $$max\thinspace \pi_i(q_i)=q_i-cq_i$$
 condierando la maximizaci\'on de los beneficios de la empresa 1 tenemos:
 $$max\thinspace \pi_1(q_1)=(\frac{a}{(q_1+q_2)^\alpha})q_1-cq_1$$
podemos emplear derivadas parciales para obtener la condici\'on de primer orden:
  $$\frac{\partial(\pi_1(q_1,q_2)}{\partial q_1} = \frac{\partial}{\partial q_1} (\frac{aq_1}{cq_1+q_2)^\alpha})$$
  derivando podemos llegamos a la expresi\'on:
  $$-a(q_1+q_2)^{-(\alpha+1)}(\alpha q_1+q_1+q_2)-c=0$$
  
Para resolver la ecuación anterior asignamos valores arbitrarios as variables (la disposici\'on a pagar superior al coste marginal,a>c). Si suponemos que las empresas producen un bien id\'entico $q_1=q_2$ podemos obterner una solución de la produci\'on con un programa de c\'alculo como R:

\begin{Schunk}
\begin{Sinput}
> a=0.9;co=0.3;alfa=0.5
> p1=function(x){(a*(2*x)^-(alfa+1))*(-alfa*x+2*x)-co}
> plot(p1) 
> o1=uniroot(p1,c(0.1,2050))
> (raiz1=o1$root)
\end{Sinput}
\begin{Soutput}
[1] 2.531251
\end{Soutput}
\end{Schunk}
\includegraphics{PUTAPRUEBA-001}

Vemos una función decreciente con el coste y nos arroja un resultado de produción $q_1=q_2=q=2.53$

\subsection{Empresas producen con $C_i(q_i)=\frac{c}{2}q_i^2$}

Suponiendo costes cuadr\'aticos podemos llegar facilmente a la expresión que maximiza los beneficios de la empresa 1:

  $$-a(q_1+q_2)^{-(\alpha+1)}(\alpha q_1+q_1+q_2)-cq_1=0$$

Resolviendo la euación anterior con la suposici\'onde que las empreses producen bienes hoogeneos obtenemos:
  
\begin{Schunk}
\begin{Sinput}
> a=0.9;co=0.3;alfa=0.5
> p2=function(x){(a*(2*x)^-(alfa+1))*(-alfa*x+2*x)-co*x}
> plot(p2) 
> o2=uniroot(p2,c(0.1,2050))
> (raiz2=o2$root)
\end{Sinput}
\begin{Soutput}
[1] 1.362841
\end{Soutput}
\end{Schunk}
\includegraphics{PUTAPRUEBA-002}

Como vemos según crece la funci\'on de coses podemos ver que las empresas con bienes homogeneos producir\'an menos 
$$q_{cq_i}>q_{\frac{c}{2}q_i^2}$$

\section{Ejercicio 3: Calcular ECN en condiciones generales} 

\subsection{Para n empresas:}

$$Q=\sum_{i=1}^nq_i=q_i+\sum_{j\neq i}^nq_j$$             
$$p'(Q)<0; p''(Q)\geq0; p(Q)=a-Q$$                                
         
Las n empresas producen lo mismo: $Q=q_i+(n-1)q_{-i}$.
                                                                                                                                                                                              La funci\'on de beneficio para cada empresa i:            
$$\pi_i=p(Q)q_i-q_ic$$

Sustituyendo:
$$\pi_i=q_i(a-(q_i+(n-1)q_i)-c)=$$
$$q_i(a-q_i-(n-1)q_{-i}-c)=$$
$$aq_i-q_i^2-(n-1)q_{-i}q_i-cq_i$$

Derivando obtenemos la CPO:
$$\partial \pi_i/\partial q_i= a-2q_i-(n-1)q_{-i}-c=0$$

Extraemos la funci\'on de reacci\'on para cada i:
$$q_i=\frac{a-(n-1)q_{-i}-c}{2}$$

En equilibrio $q_i=q_{-i}$:
$$2q_i+(n-1)q_{-i}=a-c; (n+1)q_i=a-c;$$
$$q_i=\frac{a-c}{(n+1)}$$

$$Q=nq_i \rightarrow p=a-Q=a-n(\frac{a-c}{(n+1)}$$

Cuando el n\'umero de empresas tiende a infinito el precio tiende hacia el coste marginal:
$$\lim_{n\rightarrow \infty}a-n(\frac{a-c}{(n+1)}=a-a+c;$$
$$p=c$$

\subsection{Para 2 empresas:}

$$Q=q_i+q_j$$
$$p(Q)=a-Q; p(Q)=a-(q_i+q_j)$$

Definimos la funci\'on de beneficio para i y j:
$$\pi_i(q_i,q_j)=p(Q)q_i-cq_i=q_i(a-(q_i+q_j))-cq_i$$

Haciendo la derivada parcial obtenemos la CPO:
$$\partial \pi_i/\partial q_i= a-2q_i-q_j-c=0$$

As\'i, las mejores respuestas son:
$$q_i=\frac{a-c-q_j}{2}$$
$$q_j=\frac{a-c-q_j}{2}$$

Resolviendo:
$$q_i=q_j=\frac{(a-c)}{3}$$
$$p=a-Q=a-2(\frac{a-c}{3})=a-\frac{2}{3}a+\frac{2}{3}c=$$

ya que $a>c$:
$$\frac{1}{3}a+\frac{2}{3}c>c$$

Por lo tanto, el precio ser\'a mayor que el coste marginal generando beneficios extraordinarios para las empresas.

\section{Ejercicio 4: }

El jugador con un coste marginal m\'as bajo tendr\'a como estrategia dominante fijar un precio $p=c_2-\epsilon$, de tal forma que al jugador con un coste marginal superior no le sea rentable permanecer en el mercado.

\section{Ejercicio 5: Juegos din\'amico de Cournot-Stackelberg}

En este ejercicio intentaremos mostrar en juegos dinámicos de Cournot-Stackelberg mover primero no es siempre una ventaja. Vamos a emplear un modelo similar al del Ejercicio 3, pero con costes crecientes de la forma $cq_i^3$.

\vspace{5mm}

Primero empezaremos por maximizar los beneficios de la empresa 2:

$$max\thinspace \pi_2(q_1,q_2)=aq_2-b(q_2+q_1)q_2-cq^3_2$$

obteniendo ahora la condici\'on de primer orden:

$$(\frac{\partial}{\partial q_1})(aq_2-b(q_2+q_1)q_2-cq^3_2)$$

obtenemos como soluci\'on:

$$q_2=\frac{2b+\sqrt{4b^2+12c(-q_1b+a)}}{6c}$$

Esta es la cantidad de producci\'on que maximiza los beneficios de la empresa 2. Ahora maximizaremos los beneficios de la empresa uno considerando la produci\'on de la empresa dos que maximiza sus pripios beneficios $(q_2)$:

$$max\thinspace \pi_1(q_1,q_2)=aq_1-b(q_1+\frac{2b+\sqrt{4b^2+12c(-q_1b+a)}}{6c}q_1-cq^3_1$$

realizando la derivada parcial, con respecto a $q1$, y reordenando obtenemos la siguiente ecuación:

$$a-2bq_1+\frac{2b^2}{6c}+\frac{b}{6c}\frac{q_1(a+2b(9q_1+2))}{\sqrt{q_1^2(a+4(3bq_1+b)})}-3q_1^2=0$$

Para resolver esta ecuaci\'on anterior utilizaremos el siguiente Script de R:
\begin{Schunk}
\begin{Sinput}
> a=0.6 ;b=0.2 ;co=0.4 
> p1=function(x){a-2*b*x+2*b^2/
+     (6*co)+(b/(6*co)*(x*(a+2*b*(9*x+2)))/
+ +                                        (sqrt((x^2)*(a+4*(3*b*x+b)))))-3*x^2}
> plot(p1) 
> o1=uniroot(p1,c(0.1,1))
> (raiz1=o1$root)
\end{Sinput}
\begin{Soutput}
[1] 0.4518852
\end{Soutput}
\end{Schunk}
\includegraphics{PUTAPRUEBA-003}



\begin{Schunk}
\begin{Sinput}
> p2=function(x){6*co*x-2*b-sqrt(4*b^2+12*co*(-raiz1*b+a))}
> o2=uniroot(p2,c(0.1,1))
> plot(p2)
> (raiz2=o2$root)
\end{Sinput}
\begin{Soutput}
[1] 0.8393208
\end{Soutput}
\end{Schunk}
\includegraphics{PUTAPRUEBA-004}



Hemos dados valores aleatorios a las variables $a$, $b$ y $c$ y obtenemos la raiz de $q_1$. Los valores otorgados a las variables son aleatorios pero tienen cierta coherencia. La variable $a$ (m\'aima disposici\'on a pagar) es mayor que el coste marginal $c$. Por otro lado, como $b=0,2$ la pendiente de la demanda es muy pequeña por lo que la eslasticidade precio-demanda es muy grande.

Vemos de esta manera que la produci\'on de la empresa 1 ($q_1=0.45$) es inferior a la de la empresa 2 ($Q_2=0.83$) (supuesto que las empressas son homogeneass) como queriamos demostrar.



\section{Ejercicio 6: Generalizaci\'on de juego est\'atico con informaci\'on incompleta}
Consideremos las siguientes situaci\'on:
\begin{table}[htbp]
\begin{center}
\begin{tabular}{|l|l|l|l|}
\hline
1/2 & a & b & c \\
\hline \hline
A & 4,2 & 4,2 & 4,0 \\ \hline
B & 6,6 & 0,10 & 0,0 \\ \hline

\end{tabular}
\caption{Jugador 2 de tipo x}
\label{tabla:sencilla}
\end{center}
\end{table}

\begin{table}[htbp]
\begin{center}
\begin{tabular}{|l|l|l|l|}
\hline
1/2 & a & b & c \\
\hline \hline
A & 4,2 & 4,0 & 4,3 \\ \hline
B & 6,6 & 0,0 & 0,10 \\ \hline

\end{tabular}
\caption{Jugador 2 de tipo z}
\label{tabla:sencilla}
\end{center}
\end{table}



La probabilidad de que el jugador 2 sea de tipo x es p, as\'i mismo, la pobabilidad de que sea de tipo z es (1-p).


\subsection{Los dos jugadores tienen informaci\'on incompleta y sim\'etrica (por lo tanto ambos asignan las probabilidades p y (1-p))}

Al ser un juego simult\'aneo, el conjunto de estrategias coincide con el conjunto de acciones para cada jugador:
$$S_1=A_1=[A,B]$$
$$S_2=A_2=[a,b,c]$$

Supongamos, en primer lugar, que 1 juega A. Los valores esperados de jugar sus posibles acciones para el jugador 2 son:

$$VE^2_{(a)}=2p+(1-p)2=2$$
$$VE^2_{(b)}=2p+(1-p)0=2p$$
$$VE^2_{(c)}=0p+(1-p)3=1-3p$$

Programando estas funciones en R resulta f\'acil visualizar el VE para el jugador 2:
\begin{Schunk}
\begin{Sinput}
> a=function(p){2*p+(1-p)*2}
> b=function(p){2*p+(1-p)*0}
> d=function(p){p*0+(1-p)*3}
> plot(a, ylim=c(0,3.5))
> plot(b,add=TRUE)
> plot(d,add=TRUE)
\end{Sinput}
\end{Schunk}
\includegraphics{PUTAPRUEBA-005}

Punto de corte entre d y a:
\begin{Schunk}
\begin{Sinput}
> t=function(p){(p*0+(1-p)*3)-2}
> (cortead=uniroot(t,c(0,1))$root) #punto de corte entre d y a
\end{Sinput}
\begin{Soutput}
[1] 0.3333333
\end{Soutput}
\begin{Sinput}
> 
\end{Sinput}
\end{Schunk}

Punto de corte entre a y c:
\begin{Schunk}
\begin{Sinput}
> g=function(p){(2*p+(1-p)*2)-(2*p+(1-p)*0)}
> (corteab=uniroot(g,c(0,1))$root)
\end{Sinput}
\begin{Soutput}
[1] 1
\end{Soutput}
\end{Schunk}

Dando lugar a la siguiente funci\'on de reacci\'on para el jugador 2:

$$\left\{ \begin{array}{c} b, si p<1/3\\ a, si 1/3<p<1\\b,a, si: p=1\end{array}\right. $$


Si 1 juega B:
$$VE^2_{(a)}=6p+(1-p)6=6$$
$$VE^2_{(b)}=10p+(1-p)0=10p$$
$$VE^2_{(c)}=0p+(1-p)10=1-10p$$

Graficando el VE para el jugador 2:

\begin{Schunk}
\begin{Sinput}
> a=function(p){6*p+(1-p)*6}
> b=function(p){10*p+(1-p)*0}
> d=function(p){p*0+(1-p)*10}
> plot(a, ylim=c(0,10))
> plot(b,add=TRUE)
> plot(d,add=TRUE)
\end{Sinput}
\end{Schunk}
\includegraphics{PUTAPRUEBA-008}

Puntos de corte:
\begin{Schunk}
\begin{Sinput}
> k=function(p){(p*0+(1-p)*10)-(6*p+(1-p)*6)}
> (corte1=uniroot(k,c(0,1))$root)
\end{Sinput}
\begin{Soutput}
[1] 0.4
\end{Soutput}
\end{Schunk}

Y:
\begin{Schunk}
\begin{Sinput}
> m=function(p){(10*p+(1-p)*0)-(6*p+(1-p)*6)}
> (corte2=uniroot(m,c(0,1))$root)
\end{Sinput}
\begin{Soutput}
[1] 0.6
\end{Soutput}
\end{Schunk}

Obtenemos la funci\'on de reacci\'on para el jugador 2 cuando el jugador 1 juega B:

$$\left\{ \begin{array}{c} b, si p<0.4 \\ a, si 0.4<p<0.6\\c, si: p>0.6\end{array}\right. $$

As\'i, podemos obtener los Equilibrios de Nash correspondientes a todo el espectro de posibles probabilidades subjetivas p y (1-p):

Para p<1/3; El jugador 2 juega:
  c si 1 juega A
  b si 1 juega B
  
Puesto que en esta situaci???n A es una estrategia dominante para el jugador 1, el Equilibrio de Nash ser\'ia (c,A).

Para 1/3<p<0.4; Existen 2 equilibrios de Nash: (a,A) y (b,B).

Para 0.4<p<0.6; El jugador 2 juega siempre a y el jugador 1 B; Equilibrio de Nash: (a,B).

Para p>0.6; Equilibrio de Nash: (a,A), (c,B).


\subsection{solo el jugador 1 desconoce el tipo del jugador 2}

Si 2 es de tipo x, mejor respuesta de 2:
$$MR^2_{(A)}=a,b$$
$$MR^2_{(B)}=b$$

Mejor respuesta de 1:
$$MR^1_{(a)}=B$$
$$MR^1_{(b)}=A$$
$$MR^1_{(c)}=A$$

As\'i, el Equilibrio de Nash si 2 es de tipo x es (b,A)


Si 2 es de tipo z, mejor respuesta de 2:
$$MR^2_{(A)}=c$$
$$MR^2_{(B)}=c$$

Mejor respuesta de 1:
$$MR^1_{(a)}=B$$
$$MR^1_{(b)}=A$$
$$MR^1_{(c)}=A$$

As\'i, el Equilibrio de Nash si 2 es de tipo z es (C,A)

\section{Ejercicio 7: Competencia Bertrand con informaci\'on incompleta}

Competencia Bertrand con producto homog\'eneo, es decir, $d=1$:

El coste marginal de la empresa 1 puede ser $\bar{c}$ o $\underline{c}$.
El coste marginal de la empresa 2 es c con certeza.
Se cumple:
$$0<\underline{c}<c<\bar{c}$$

La empresa 2 conoce la distribuci\'on de probabilidad del coste marginal de la empresa 1, es decir, sabe que el coste la empresa 1 es $\underline{c}$ con probabilidad $\gamma$; $0<\gamma<1$ y $\bar{c}$ con probabilidad $(1-\gamma)$.

La funci\'on de demanda residual de la empresa es $q_i(p_i,p_j)=a-p_i+dp_j$. Sabiendo que $d=1$ para productos homog\'eneos (m\'aximo grado de substituci\'on) obtenemos:
$$q_i(p_i,p_j)=a-p_i+p_j$$ $$i,j=1,2$$ $$i\neq j$$

La estrategia \'optima para cada empresa es decidir el precio que constituya la mejor respuesta apra cada uno de los tipos  en los que podr\'ia encarnarse. As\'i, sus mejores respuestas provendr\'an de la maximizaci\'on de sus funciones de beneficios.

Para 1 si su tipo es $\underline{c}$:
$$\pi_1(p_1,p_2)=(p_1-\underline{c})(a-p_1+p_2)$$

CPO:
$$p_1(\underline{c};p_2)=\frac{a+\underline{c}+dp_2}{2}$$

Para 1 si su tipo es $\bar{c}$:
$$\pi_1(p_1,p_2)=(p_1-\bar{c})(a-p_1+p_2)$$

CPO:
$$p_1(\bar{c};p_2)=\frac{a+\bar{c}+dp_2}{2}$$

Para la empresa 2, la funci\'on de beneficios esperados ser\'a:
$$E[\pi_2]=(p_2-c)[\gamma(a-p_2+p_1(\underline{c})+(1-\gamma)(a-p_2+p_1(\bar{c}))]$$

El precio \'optimo que fijar\'a la empresa 2 de acuerdo con sus creencias sobre los costes de la empresa 1 ser\'a:
$$p_2(c;p_1(\underline{c}),p_1(\bar{c}))=\frac{a+c+(\gamma p_1(\underline{c})+(1-\gamma)p_1(\bar{c}))}{2}$$

Resolviendo, el EB de el duopolio con bien homog\'eneo e informaci\'on incompleta de Bertrand es el perfil de estrategias:

$${(p_1^*(\underline{c}),p_1^*(\bar{c}),p_2^*)}=$$
\hspace{2cm}
$${(\frac{6a-(4-(1+\gamma))\underline{c}+(1-\gamma)\bar{c}+c}{6},\frac{6a+(4-\gamma)\bar{c}+\gamma \underline{c}+2c}{6},\frac{3a+2c+(\gamma \underline{c}+(1-\gamma)\bar{c})}{3})}$$




\section{Ejercicio 8: Subastas}
\subsection{Sobre cerrado segundo precio.}
Investigar qu\'e sueder\'ia con la estrategia $b_i^`(v_i)<v_i$, es decir, cuando el jugador puja una cantidad de dinero superior a su valoraci\'on.

Se pueden presentar los siguientes casos:

1.-Gana la subasta si $b'_i(v_i)>\bar{b}$, obteniendo un pago de $v_i-b'_i(v_i)<0$.

2.-Gana la subasta si $b'_i(v_i)>\bar{b}$, pero obtiene un pago negativo de $v_i-b'_i(v_i)<0$. Esto sucede cuando $b'_i(v_i)-\bar{b}$ > $v_i-\bar{b}$.

3.-Pierde la subasta si $b'_i(v_i)<\bar{b}$.

\subsection{Sobre cerrado al primer precio}

En este tipo de subastas, el ganador es aquel jugador que ofreciese la puja m\'as alta. El pago ser\'a su ropia oferta. Adem\'as, los jugadores no pueden ver las pujas de los dem\'as.

Calcular la MRi y obtener EBN.

Conjunto de acciones posibles para cada jugador: $i=1,..,n$. Se corresponde con el conjunto de pujas que puede ofertar.
El conjunto de acciones y de estrategias coincide al ser un juego simult\'aneo:
$$A_i=S_i=[0,+\infty)$$

Los tipos de cada jugador i son las distintas valoraciones que puede tener del objeto subastado:
$$T_i=[0,\bar{v}]$$

Cada i cree que el resto de valoraciones $v_j$ est\'a uniformemente distribuida en $[0,\bar{v}]$.

El pago de cada jugador es su beneficio esperado.

El objetivo de cada jugador es maximizar el beneficio.

En esta situaci\'on, el EB ser\'a el perfil de pujas:
$${(b_1^*,...,b_n^*)}={(\frac{n-1}{n}v_1,...,\frac{n-1}{n}v_n)}$$

No hay estrategia dominante, cada jugador debe determinar su mejor respuesta frente a la estrategia de los dem\'as.

Definimos la probabilidad de que i puje por debajo de b:
$$Pr(B_i\leq b)=Pr(v_i\leq B^{-1}(b))=F(B^{-1}(b))=\frac{B^{-1(b)}}{\bar{v}}$$

Todo jugador distinto de i puede saber con qu\'probabilidad ganar\'a la subasta conociendo la probabilidad anterior, esto se debe a que cuando puja b, conoce la probabilidad de que i puje menos que b.

La probabilidad de que todos los jugadores menos el jugador 1 pujen menos que b es:
$$Pr(B_i\leq b, \forall i=2,3,...,n)=(\frac{B^{-1(b)}}{\bar{v}})^{n-1}$$
 Por lo tanto, esta es la probabilidad de que 1 gane la subasta puando b.

La utilidad esperada de 1 pujando b es:
$$E[u_1]=(v_1-b)(\frac{B^{-1(b)}}{\bar{v}})^{n-1}$$

Teniedo en cuenta que si gana obtiene $v_1-b$, y si pierde obtiene 0.

Maximizando la utilidad esperada obtenemos la siguiente CPO:
$$\partial E[u_1]/\partial b=0=(-1)(\frac{B^{-1(b)}}{\bar{v}})^{n-1}+(v_1-b)(n-1)(\frac{B^{-1(b)}}{\bar{v}})^{n-2}\frac{1}{\bar{v}}\frac{dB^{-1}(b)}{db}$$

Reescribiendo:
$$(\frac{B^{-1(b)}}{\bar{v}})^{n-1}=(v_1-b)(n-1)(\frac{B^{-1(b)}}{\bar{v}})^{n-2}\frac{1}{\bar{v}}\frac{dB^{-1}(b)}{db}$$

Simplificando:
$$B^{-1}(b)\frac{dB}{dv}(B^{-1}(b))=(v_1-b)(n-1)$$

Asumiendo todos los jugadores sim\'etricos: $B^{-1}(b)=v_1$ y $b=B(v_1)$; As\'i:
$$\frac{dB}{dv}(v)=(1-\frac{B(v)}{v})(n-1)$$

Esta expresi\'on es una ecuaci\'on diferencial ordinaria de primer orden. Fijando la condici\'on $B(0)=0$ (La puja de un jugador con valoraci\'on nula es igual a 0) se obtiene:
$$B(v)=\frac{n-1}{n}v$$

Esta expresi\'on denota la relaci\'on existente entre la puja de cada jugador y su valoraci\'on. Como $\frac{n-1}{n}<1$, los jugadores pujan por debajo de sus verdaderas valoraciones en el EB.



\section{Ejercicio 9: Provisi\'on de bien p\'ublico}
\subsection{Modificaci\'on de pagos para que ninguno de los dos agentes tenga estrategia dominante.}

En este ejercicio en un caso de provisión de un bien p\'ublico consideramos: Una familia formada por los individuos 1 y 2 los cuales deben decidir simultaneamente y puede decidir contribuir o no contribuir, $\{C,N\}_i, i=1,2$. El bien p\'ublico se suministrar\'a si alguno de los individuos o \'ambos eligen contribuir. Contribuir a financiar el bien p\'ublio le supone al jugador 1 un coste de $c_1$ y al individuo 2 un coste que puede ser $\overline{c_2}$ o $\underline{c_2}$. El individuo 2 los costes que le supone al individuo 1 pero este \'ultimo no conoce los costes para le individuo 2(lo único que sabe es que ouede ser $\overline{c_2}$ con probabilidad $\frac{1}{3}$ o $\underline{c_2}$ con probabilidad $\frac{2}{3}$).

\hspace{2cm}

La tarea propuesta es considerando el caso de los apuntes, modificar los pagos (los cuales mostraremos tabulados) para que el individuo 2 no posea una estrategia dominante como es la de contribuir. Para eso modificamos los pagos de la forma:

\begin{table}[htbp]
\begin{center}
\begin{tabular}{|l|r|r|}
\hline
\backslashbox{1}{2} & C & N \\
\hline
C & 1-$c_1$,1-$\underline{c_2}$ & 1-$c_1$,1\\
\hline
N & 1,1-$\underline{c_2}$ & 0,0\\
\hline
\end{tabular}
\caption{Jugador 2 del tipo $\underline{c_2}$ con probabilidad $\frac{1}{3}$ }
\label{tabla:sencilla}
\end{center}
\end{table}

\begin{table}[htbp]
\begin{center}
\begin{tabular}{|l|r|r|}
\hline
\backslashbox{1}{2} & C & N \\
\hline
C & $1-c_1,1-\underline{c_2}$ & 1-$c-1$,1\\
\hline
N & 1,1-$\overline{c_2}$ & 0,2\\
\hline
\end{tabular}
\caption{Jugador 2 del tipo $\overline{c_2}$ con probabilidad $\frac{2}{3}$ }
\label{tabla:sencilla}
\end{center}
\end{table}



El principal cambio realizado es el pago de 2 cuando es de tipo $\overline{c_2}$ y 1 elige no contribuir. Con este cambio el individuo 2 cunando es de tipo $\overline{c_2}$ tendr\'a la estrategia dominante de no contribuir.
Con estes pagos si analizamos los valores esperados con la finalidad de obtener la mejor respuestas del individuo 2 vemos que:

\hspace{0.5cm} 

Si el individuo 2 es de tipo $\overline{c_2}$:

$$E(u_2,\overline{c_2},1C)\Longrightarrow (1-\overline{c_2})\gamma+(1-\gamma)$$
$$E(u_2,\overline{c_2},1N)\Longrightarrow (1-\overline{c_2})\gamma+2(1-\gamma)$$

Si el individuo 2 es de tipo $\underline{c_2}$:

$$E(u_2,\underline{c_2},1C)\Longrightarrow (1-\underline{c_2})\gamma+(\gamma-1)$$
$$E(u_2,\underline{c_2},1N)\Longrightarrow (1-\underline{c_2})\gamma+0$$

Vemos con esto los valores esperados el individuo 2 tendr\'a una estrategia dominante dependiente de su tipo.

Ahora analizaremos los pagos esperados del individuo 1, que asigna una probabilidad $\mu$ a que le individuo 2 sea de tipo $\bar{c_2}$ y elija N y una probabbilidad $\lambda$ a que el ndividuo 2 sea de tipo $\underline{c_2}$ y elija C.

Si el individuo 1 elige C:

$$E(u_1,C)=\frac{2}{3}(1-\mu)(1-c_1)+\frac{1}{3}\lambda(1-c_1)+\frac{2}{3}(1-c_1)+\frac{1}{3}(1-\lambda)(1-c_1)=1-c_1$$

Si el individuo q elige N:

$$E(u_1,N)=\frac{2}{3}(1-\mu)(1-c_1)+\frac{1}{3}\lambda$$

Las mejores respestas del individuo 1 son:

$$\left\{ \begin{array}{c} C si c_1<\frac{1}{3}(2\mu-\lambda+1) \\ N si c_1>\frac{1}{3}(2\mu-\lambda+1)\end{array}\right. $$


Por lo tanto tenemos que si $c$ es suficientemente bajo

-Si el individuo 1 toma la estrategia  i$S_1^*=C$ el individuo 2 tiene las estrategias $S_2^*(\underline{c_2}=C$ $$S_2^*(\overline{c_2}=N$$. Si $S_1^*=N$ 

-Si el individuo 1 toma la estrategia  i$S_1^*=N$ el individuo 2 tiene las estrategias $S_2^*(\underline{c_2}=C$ $$S_2^*(\overline{c_2}=N$$. Si $S_1^*=N$ 


Y si $c$ es suficientemente grande y el individuo 2 es de tipo $\overline{c_2}$ ninguno de los individuos optar\'a por contribuir.



\subsection{Generalizar el modelo con el sistema de creencias $(p,1-p)$.}

Si generalizamos a una probabilidade p nos encontramos con los siguientes pagos:

\begin{table}[htbp]
\begin{center}
\begin{tabular}{|l|r|r|}
\hline
\backslashbox{1}{2} & C & N \\
\hline
C & 1-$c_1$,1-$\underline{c_2}$ & 1-$c_1$,1\\
\hline
N & 1,1-$\underline{c_2}$ & 0,0\\
\hline
\end{tabular}
\caption{Jugador 2 del tipo $\underline{c_2}$ con probabilidad (1-p)}
\label{tabla:sencilla}
\end{center}
\end{table}

\begin{table}[htbp]
\begin{center}
\begin{tabular}{|l|r|r|}
\hline
\backslashbox{1}{2} & C & N \\
\hline
C & $1-c_1,1-\underline{c_2}$ & 1-$c-1$,1\\
\hline
N & 1,1-$\overline{c_2}$ & 0,0\\
\hline
\end{tabular}
\caption{Jugador 2 del tipo $\overline{c_2}$ con probabilidad p }
\label{tabla:sencilla}
\end{center}
\end{table}

Los pagos esperados de 2, considerando que conoce su tipo ser\'an
$$E(u_2,\overline{c_2},1C)\Longrightarrow (1-\overline{c_2})\gamma+(1-\gamma)$$
$$E(u_2,\overline{c_2},1N)\Longrightarrow (1-\overline{c_2})\gamma$$

El cuel tendr\'an como estrategia dominante contribuir.

Los pagos esperados de 1, asignando este una probabilidad p a que 2 sea de tipo $\overline{c_2}$ ser\'an:

Si el individuo 1 elige C:

$$E(u_1,C)=p(1-\mu)(1-c_1)+(1-p)\lambda(1-c_1)+p(1-c_1)+(1-p)(1-\lambda)(1-c_1)=1-c_1$$

Si el individuo 1 elige N:

$$E(u_1,N)=p(1-\mu)(1-c_1)+(1-p)\lambda$$

as mejores respestas del individuo 1 son:

$$\left\{ \begin{array}{c} C si c_1<(1-p)(2\mu-\lambda+1) \\ N si c_1>(1-p)(2\mu-\lambda+1)\end{array}\right. $$

Como vemos, la mejor respuesta del individuo uno depender\'a del valor que este asigne a p y as\'i conocer que costes son para \'el suficientemente pequeños y cuales suficientemente elevados.
\end{document}
