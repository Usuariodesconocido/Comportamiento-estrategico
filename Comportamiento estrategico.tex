%%%%%%%%%%%%%%%%%%%%%%%%%%%%%%%%%%%%%%%%%%%%%%%%%%%%%%%%%%%%%%%%%%%%%%%%%%%
%
% Plantilla para un art�culo en LaTeX en espa�ol.
%
%%%%%%%%%%%%%%%%%%%%%%%%%%%%%%%%%%%%%%%%%%%%%%%%%%%%%%%%%%%%%%%%%%%%%%%%%%%

\documentclass{article}

% Esto es para poder escribir acentos directamente:
\usepackage[latin1]{inputenc}
% Esto es para que el LaTeX sepa que el texto est� en espa�ol:
\usepackage[spanish]{babel}

% Paquetes de la AMS:
\usepackage{amsmath, amsthm, amsfonts}

% Teoremas
%--------------------------------------------------------------------------
\newtheorem{thm}{Teorema}[section]
\newtheorem{cor}[thm]{Corolario}
\newtheorem{lem}[thm]{Lema}
\newtheorem{prop}[thm]{Proposici�n}
\theoremstyle{definition}
\newtheorem{defn}[thm]{Definici�n}
\theoremstyle{remark}
\newtheorem{rem}[thm]{Observaci�n}

% Atajos.
% Se pueden definir comandos nuevos para acortar cosas que se usan
% frecuentemente. Como ejemplo, aqu� se definen la R y la Z dobles que
% suelen representar a los conjuntos de n�meros reales y enteros.
%--------------------------------------------------------------------------

\def\RR{\mathbb{R}}
\def\ZZ{\mathbb{Z}}

% De la misma forma se pueden definir comandos con argumentos. Por
% ejemplo, aqu� definimos un comando para escribir el valor absoluto
% de algo m�s f�cilmente.
%--------------------------------------------------------------------------
\newcommand{\abs}[1]{\left\vert#1\right\vert}

% Operadores.
% Los operadores nuevos deben definirse como tales para que aparezcan
% correctamente. Como ejemplo definimos en jacobiano:
%--------------------------------------------------------------------------
\DeclareMathOperator{\Jac}{Jac}

%--------------------------------------------------------------------------
\title{Entrega Comportamiento Estrat\'egico}
\author{Pablo Coello Pulido\\
}

\begin{document}
\maketitle

\abstract{Tareas para entregar de la asignatura An\'alisis del Comportamiento Estrat\'egico. M\'aster en Econom\'ia.}

\section{Ejercicio 1:}

En el siguiente Juego a la Cournot la demanda de mercado que enfrentan n empresas es estrictamente convexa con la forma
$p(q)=\frac{a}{Q^\alpha}$, donde $Q=\sum_{i=1}^n q_i$. Los consumidores se asumen pasivos y vienen descritos por $p(Q)=a-bQ$.
\subsection{Empresas producen con $C_i(q_i)=cq_i$}
Si consideramos que el n\'umero de empresas \textbf{n} es 2, podemos calcular las cantidades que maximizan sus beneficios como:
 $$max\thinspace \pi_i(q_i)=q_i-cq_i$$
 condierando la maximizaci\'on de los beneficios de la empresa 1 tenemos:
 $$max\thinspace \pi_1(q_1)=(\frac{a}{(q_1+q_2)^\alpha})q_1-cq_1$$
podemos emplear derivadas parciales para obtener la condici\'on de primer orden:
  $$\frac{\partial(\pi_1(q_1,q_2)}{\partial q_1} = \frac{\partial}{\partial q_1} (\frac{aq_1}{cq_1+q_2)^\alpha})$$
  derivando podemos llegamos a la expresi\'on:
  $$-a\alpha q_1(q_1+q_2)^{-(\alpha+1)}+a(q_1+q_2)^{-\alpha}-c=0$$
  operando matematicamente podemos obtener $q_2=q_2(q_1)$:
  $$q_2=\sqrt{\frac{\frac{-c}{a}}{\alpha q_1-1}}^{\frac{1}{\alpha-1}}$$
  
  
\subsection{Empresas producen con $C_i(q_i)=\frac{c}{2}q_i^2$}                                                                                                                                                                                                                                                                                                                                                                                                                                                                                                                                                                                                                                                                                                                                                                                                                                                                                                                                                                                                                                                                                                                                                                                                                                                                                                                                                                    

\section{Ejercicio 2:}                     
\subsection{Para n empresas:}

$$Q=\sum_{i=1}^nq_i=q_i+\sum_{j\neq i}^nq_j$$             
$$p'(Q)<0; p''(Q)\geq0; p(Q)=a-Q$$                                
         
Las n empresas producen lo mismo: $Q=q_i+(n-1)q_{-i}$
                                                                                                                                                                                                           
$$\pi_i=p(Q)q_i-q_ic$$
$$\pi_i=q_i(a-(q_i+(n-1)q_i)-c)=$$
$$q_i(a-q_i-(n-1)q_{-i}-c)=$$
$$aq_i-q_i^2-(n-1)q_{-i}q_i-cq_i$$

$$\partial \pi_i/\partial q_i= a-2q_i-(n-1)q_{-i}-c=0$$
$$q_i=\frac{a-(n-1)q_{-i}-c}{2}$$

En equilibrio $q_i=q_{-i}$:
$$2q_i+(n-1)q_{-i}=a-c; (n+1)q_i=a-c;$$
$$q_i=\frac{a-c}{(n+1)}$$

$$Q=nq_i \rightarrow p=a-Q=a-n(\frac{a-c}{(n+1)}$$

Cuando el n\'umero de empresas tiende a infinito el precio tiende hacia el coste marginal:
$$\lim_{n\rightarrow \infty}a-n(\frac{a-c}{(n+1)}=a-a+c;$$
$$p=c$$

\subsection{Para 2 empresas:}

$$Q=q_i+q_j$$
$$p(Q)=a-Q; p(Q)=a-(q_i+q_j)$$

$$\pi_i(q_i,q_j)=p(Q)q_i-cq_i=q_i(a-(q_i+q_j))-cq_i$$
$$\partial \pi_i/\partial q_i= a-2q_i-q_j-c=0$$
As\'i, las mejores respuestas son:
$$q_i=\frac{a-c-q_j}{2}$$
$$q_j=\frac{a-c-q_j}{2}$$
Resolviendo:
$$q_i=q_j=\frac{(a-c)}{3}$$
$$p=a-Q=a-2(\frac{a-c}{3})=a-\frac{2}{3}a+\frac{2}{3}c=$$
ya que $a>c$:
$$\frac{1}{3}a+\frac{2}{3}c>c$$
Por lo tanto, el precio ser\'a mayor que el coste marginal generando beneficios extraordinarios para las empresas.

\section{Ejercicio 3:}
\section{Ejercicio 4:}
\section{Ejercicio 5: Competencia Stackelberg-Bertrand con producto diferenciado.}
$$p_1(q_1,q_2)=1-q_1-dq_2; d\in(-1,1)$$
$$p_2(q_1,q_2)=1-q_2-dq_1; d\in(-1,1)$$

N\'otese que d cuantifica el grado de diferenciaci\'on. Cuando $d=1$, $q_1$ y $q_2$ son sustitutos perfectos. Cuando  $d=-1$, $q_1$ y $q_2$ son complementarios perfectos.

$$\max\pi_2(p_1,p_2)=p_2(\frac{a}{1+d}-\frac{1}{1-d^2}p_2+\frac{1}{1-d^2}p_1)$$

Despejando de la CPO:

$$p_2(p_1)=\frac{a(1-d)}{2}+\frac{d}{2}p_1$$

De esta expresi\'on se deduce que los precios son complementarios estrat\'egicos.

Sustituyendo:
$$\max\pi_1(p_1,p_2(p_1))=p_1(\frac{a}{1+d}-\frac{1}{1-d^2}p_1+\frac{d}{1-d^2}p_2(p_1))$$

Operando se obtiene el Equilibrio de Nash:
$$ENPS:(p_1^s,p_2^s)=(\frac{a(2-d-d^2}{2(2-d^2)},\frac{a(4-2d-3d^2+d^3}{4(2-d^2)})$$

De donde podemos obtener las cantidades:
$$q_1^s=\frac{a(2+d)}{4(1+d)}$$
$$q_2^s=\frac{a(1-d)(4+2d-d^2)}{4(4-3d^2+d^4)}$$

Y los beneficios:
$$\pi_1^s=\frac{a^2(2-d-d^2)^2}{8(2-3d^2+d^4)}$$
$$\pi_2^s=\frac{a^2(4-2d-3d^2+d^3)^2}{16(1-d^2)(2-d^2)^2}$$

Se puede ver que $p_2^s-p_1^s=-\frac{a(1-d)d^2}{4(2-d^2)}<0$; por lo tanto, el precio elegido por la empresa seguidora es menor que el de la empresa l\'ider.


Esto se debe a que la empresa seguidora pone un precio inferior al de la empresa l\'ider para as\'i ganar cuota de mercado:
$$\pi_2^s-\pi_1^s=\frac{a^2d^3(4-d-3d^2)}{16(1+d)(2-d^2)^2}>0$$
$$\pi_2^2>\pi_1^s   \forall   0<d<1$$

Es decir, existe ventaja a la hora de actuar como seguidor cuando los bienes son sustitutos en cierto grado.

\section{Ejercicio 6:}
\section{Ejercicio 7:Competencia Bertrand}

Competencia Bertrand con producto homog\'eneo, es decir, $d=1$:

El coste marginal de la empresa 1 puede ser $\bar{c}$ o $\underline{c}$.
El coste marginal de la empresa 2 es c con certeza.
Se cumple:
$$0<\underline{c}<c<\bar{c}$$

La empresa 2 conoce la distribuci\'on de probabilidad del coste marginal de la empresa 1, es decir, sabe que el coste la empresa 1 es $\underline{c}$ con probabilidad $\gamma$; $0<\gamma<1$ y $\bar{c}$ con probabilidad $(1-\gamma)$.

La funci\'on de demanda residual de la empresa es $q_i(p_i,p_j)=a-p_i+dp_j$. Sabiendo que $d=1$ para productos homog\'eneos (m\'aximo grado de substituci\'on) obtenemos:
$$q_i(p_i,p_j)=a-p_i+p_j$$ $$i,j=1,2$$ $$i\neq j$$

La estrategia \'optima para cada empresa es decidir el precio que constituya la mejor respuesta apra cada uno de los tipos  en los que podr\'ia encarnarse. As\'i, sus mejores respuestas provendr\'an de la maximizaci\'on de sus funciones de beneficios.

Para 1 si su tipo es $\underline{c}$:
$$\pi_1(p_1,p_2)=(p_1-\underline{c})(a-p_1+p_2)$$

CPO:
$$p_1(\underline{c};p_2)=\frac{a+\underline{c}+dp_2}{2}$$

Para 1 si su tipo es $\bar{c}$:
$$\pi_1(p_1,p_2)=(p_1-\bar{c})(a-p_1+p_2)$$

CPO:
$$p_1(\bar{c};p_2)=\frac{a+\bar{c}+dp_2}{2}$$

Para la empresa 2, la funci\'on de beneficios esperados ser\'a:
$$E[\pi_2]=(p_2-c)[\gamma(a-p_2+p_1(\underline{c})+(1-\gamma)(a-p_2+p_1(\bar{c}))]$$

El precio \'optimo que fijar\'a la empresa 2 de acuerdo con sus creencias sobre los costes de la empresa 1 ser\'a:
$$p_2(c;p_1(\underline{c}),p_1(\bar{c}))=\frac{a+c+(\gamma p_1(\underline{c})+(1-\gamma)p_1(\bar{c}))}{2}$$

Resolviendo, el EB de el duopolio con bien homog\'eneo e informaci\'on incompleta de Bertrand es el perfil de estrategias:

$${(p_1^*(\underline{c}),p_1^*(\bar{c}),p_2^*)}={(\frac{6a-(4-(1+\gamma))\underline{c}+(1-\gamma)\bar{c}+c}{6},\frac{6a+(4-\gamma)\bar{c}+\gamma \underline{c}+2c}{6},\frac{3a+2c+(\gamma \underline{c}+(1-\gamma)\bar{c})}{3})}$$









\section{Ejercicio 8:Subastas}
\subsection{Sobre cerrado segundo precio.}
Investigar qu\'e sueder\'ia con la estrategia $b_i^`(v_i)<v_i$.
\subsection{Sobre cerrado al primer precio}

En este tipo de subastas, el ganador es aquel jugador que ofreciese la puja m\'as alta. El pago ser\'a su ropia oferta. Adem\'as, los jugadores no pueden ver las pujas de los dem\'as.

Calcular la MRi y obtener EBN.

Conjunto de acciones posibles para cada jugador: $i=1,..,n$. Se corresponde con el conjunto de pujas que puede ofertar.
El conjunto de acciones y de estrategias coincide al ser un juego simult\'aneo:
$$A_i=S_i=[0,+\infty)$$

Los tipos de cada jugador i son las distintas valoraciones que puede tener del objeto subastado:
$$T_i=[0,\bar{v}]$$

Cada i cree que el resto de valoraciones $v_j$ está uniformemente distribuida en $[0,\bar{v}]$.

El pago de cada jugador es su beneficio esperado.

El objetivo de cada jugador es maximizar el beneficio.

En esta situaci\'on, el EB ser\'a el perfil de pujas:
$${(b_1^*,...,b_n^*)}={(\frac{n-1}{n}v_1,...,\frac{n-1}{n}v_n)}$$

No hay estrategia dominante, cada jugador debe determinar su mejor respuesta frente a la estrategia de los dem\'as.

Definimos la probabilidad de que i puje por debajo de b:
$$Pr(B_i\leq b)=Pr(v_i\leq B^{-1}(b))=F(B^{-1}(b))=\frac{B^{-1(b)}}{\bar{v}}$$

Todo jugador distinto de i puede saber con qu\'probabilidad ganar\'a la subasta conociendo la probabilidad anterior, esto se debe a que cuando puja b, conoce la probabilidad de que i puje menos que b.

La probabilidad de que todos los jugadores menos el jugador 1 pujen menos que b es:
$$Pr(B_i\leq b, \forall i=2,3,...,n)=(\frac{B^{-1(b)}}{\bar{v}})^{n-1}$$
 Por lo tanto, esta es la probabilidad de que 1 gane la subasta puando b.

La utilidad esperada de 1 pujando b es:
$$E[u_1]=(v_1-b)(\frac{B^{-1(b)}}{\bar{v}})^{n-1}$$

Teniedo en cuenta que si gana obtiene $v_1-b$, y si pierde obtiene 0.

Maximizando la utilidad esperada obtenemos la siguiente CPO:
$$\partial E[u_1]/\partial b=0=(-1)(\frac{B^{-1(b)}}{\bar{v}})^{n-1}+(v_1-b)(n-1)(\frac{B^{-1(b)}}{\bar{v}})^{n-2}\frac{1}{\bar{v}}\frac{dB^{-1}(b)}{db}$$

Reescribiendo:
$$(\frac{B^{-1(b)}}{\bar{v}})^{n-1}=(v_1-b)(n-1)(\frac{B^{-1(b)}}{\bar{v}})^{n-2}\frac{1}{\bar{v}}\frac{dB^{-1}(b)}{db}$$

Simplificando:
$$B^{-1}(b)\frac{dB}{dv}(B^{-1}(b))=(v_1-b)(n-1)$$

Asumiendo todos los jugadores sim\'etricos: $B^{-1}(b)=v_1$ y $b=B(v_1)$; As\'i:
$$\frac{dB}{dv}(v)=(1-\frac{B(v)}{v})(n-1)$$

Esta expresi\'on es una ecuaci\'on diferencial ordinaria de primer orden. Fijando la condici\'on $B(0)=0$ (La puja de un jugador con valoraci\'on nula es igual a 0) se obtiene:
$$B(v)=\frac{n-1}{n}v$$

Esta expresi\'on denota la relaci\'on existente entre la puja de cada jugador y su valoraci\'on. Como $\frac{n-1}{n}<1$, los jugadores pujan por debajo de sus verdaderas valoraciones en el EB.

\section{Ejercicio 9:Provisi\'on de bien p\'ublico}
\subsection{Modificaci\'on de pagos para que ninguno de los dos agentes tenga estrategia dominante.}
\subsection{Generalizar el modelo con el sistema de creencias $(p,1-p)$.}
\section{Ejercicio 10:}



























































% Bibliograf�a.
%-----------------------------------------------------------------
\begin{thebibliography}{99}

\bibitem{Cd94} Autor, \emph{T�tulo}, Revista/Editor, (a�o)

\end{thebibliography}

\end{document}