%%%%%%%%%%%%%%%%%%%%%%%%%%%%%%%%%%%%%%%%%%%%%%%%%%%%%%%%%%%%%%%%%%%%%%%%%%%
%
% Plantilla para un art�culo en LaTeX en espa�ol.
%
%%%%%%%%%%%%%%%%%%%%%%%%%%%%%%%%%%%%%%%%%%%%%%%%%%%%%%%%%%%%%%%%%%%%%%%%%%%

\documentclass{article}

% Esto es para poder escribir acentos directamente:
\usepackage[latin1]{inputenc}
% Esto es para que el LaTeX sepa que el texto est� en espa�ol:
\usepackage[spanish]{babel}

% Paquetes de la AMS:
\usepackage{amsmath, amsthm, amsfonts}

% Teoremas
%--------------------------------------------------------------------------
\newtheorem{thm}{Teorema}[section]
\newtheorem{cor}[thm]{Corolario}
\newtheorem{lem}[thm]{Lema}
\newtheorem{prop}[thm]{Proposici�n}
\theoremstyle{definition}
\newtheorem{defn}[thm]{Definici�n}
\theoremstyle{remark}
\newtheorem{rem}[thm]{Observaci�n}

% Atajos.
% Se pueden definir comandos nuevos para acortar cosas que se usan
% frecuentemente. Como ejemplo, aqu� se definen la R y la Z dobles que
% suelen representar a los conjuntos de n�meros reales y enteros.
%--------------------------------------------------------------------------

\def\RR{\mathbb{R}}
\def\ZZ{\mathbb{Z}}

% De la misma forma se pueden definir comandos con argumentos. Por
% ejemplo, aqu� definimos un comando para escribir el valor absoluto
% de algo m�s f�cilmente.
%--------------------------------------------------------------------------
\newcommand{\abs}[1]{\left\vert#1\right\vert}

% Operadores.
% Los operadores nuevos deben definirse como tales para que aparezcan
% correctamente. Como ejemplo definimos en jacobiano:
%--------------------------------------------------------------------------
\DeclareMathOperator{\Jac}{Jac}

%--------------------------------------------------------------------------
\title{Entrega Comportamiento Estrat\'egico}
\author{Pablo Coello Pulido\\
}

\begin{document}
\maketitle

\abstract{Tareas para entregar de la asignatura An\'alisis del Comportamiento Estrat\'egico. M\'aster en Econom\'ia.}

\section{Ejercicio 1:}

Aqu� va el texto.
\begin{equation}\label{eq:area}
  S = \pi r^2
\end{equation}
Uno puede referirse a ecuaciones as�: ver ecuaci�n (\ref{eq:area}).
Tambi�n se pueden mencionar secciones de la misma forma: ver secci�n
\ref{sec:nada}. O citar algo de la bibliograf�a: \cite{Cd94}.

\subsection{Subsection}\label{sec:nada}

M�s texto.

\subsubsection{Subsubsection}\label{sec:nada2}

M�s texto.
\section{Ejercicio 2:}
\subsection{Para n empresas:}

$$Q=\sum_{i=1}^nq_i=q_i+\sum_{j\neq i}^nq_j$$
$$p'(Q)<0; p''(Q)\geq0; p(Q)=a-Q$$

Las n empresas producen lo mismo: $Q=q_i+(n-1)q_{-i}$

$$\pi_i=p(Q)q_i-q_ic$$
$$\pi_i=q_i(a-(q_i+(n-1)q_i)-c)=$$
$$q_i(a-q_i-(n-1)q_{-i}-c)=$$
$$aq_i-q_i^2-(n-1)q_{-i}q_i-cq_i$$

$$\partial \pi_i/\partial q_i= a-2q_i-(n-1)q_{-i}-c=0$$
$$q_i=\frac{a-(n-1)q_{-i}-c}{2}$$

En equilibrio $q_i=q_{-i}$:
$$2q_i+(n-1)q_{-i}=a-c; (n+1)q_i=a-c;$$
$$q_i=\frac{a-c}{(n+1)}$$

$$Q=nq_i \rightarrow p=a-Q=a-n(\frac{a-c}{(n+1)}$$

Cuando el n\'umero de empresas tiende a infinito el precio tiende hacia el coste marginal:
$$\lim_{n\rightarrow \infty}a-n(\frac{a-c}{(n+1)}=a-a+c;$$
$$p=c$$

\subsection{Para 2 empresas:}

$$Q=q_i+q_j$$
$$p(Q)=a-Q; p(Q)=a-(q_i+q_j)$$

$$\pi_i(q_i,q_j)=p(Q)q_i-cq_i=q_i(a-(q_i+q_j))-cq_i$$
$$\partial \pi_i/\partial q_i= a-2q_i-q_j-c=0$$
As\'i, las mejores respuestas son:
$$q_i=\frac{a-c-q_j}{2}$$
$$q_j=\frac{a-c-q_j}{2}$$
Resolviendo:
$$q_i=q_j=\frac{(a-c)}{3}$$
$$p=a-Q=a-2(\frac{a-c}{3})=a-\frac{2}{3}a+\frac{2}{3}c=$$
$$$$

\section{Ejercicio 3:}
\section{Ejercicio 4:}
\section{Ejercicio 5:}
\section{Ejercicio 6:}
\section{Ejercicio 7:}
\section{Ejercicio 8:}
\section{Ejercicio 9:}
\section{Ejercicio 10:}
\section{Ejercicio 11:}



% Bibliograf�a.
%-----------------------------------------------------------------
\begin{thebibliography}{99}

\bibitem{Cd94} Autor, \emph{T�tulo}, Revista/Editor, (a�o)

\end{thebibliography}

\end{document}